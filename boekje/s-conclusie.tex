% !TeX spellcheck = nl_NL
\chapter{Conclusie}
Deze thesis heeft als einddoel het realiseren van een autonome quadcopter die in staat is om zijn omgeving in kaart te brengen. Het einddoel is dus in twee delen op te splitsen. In het eerste deel moet de stabiliteit van het autonome platform verbeterd worden door de sturing van de quadcopter te optimaliseren. De conclusies van dit deel worden verder besproken in sectie \ref{sec:concOpt}. In het tweede deel moet een gepast SLAM algoritme gezocht worden en moet het platform worden uitgebreid om omgevingsmapping mogelijk te maken. De conclusie met betrekking tot dit deel worden verder besproken in \ref{sec:concSLAM}. Om dit hoofdstuk af te sluiten worden enkele suggesties opgesomd voor toekomstige werken.

\section{Optimalisatie} \label{sec:concOpt}
Er zijn verschillende stappen ondernomen om het vlieggedrag van het platform stabieler te maken. Deze stappen worden hieronder overlopen.

\npar Het platform gebruikt optical flow om drift te detecteren en vervolgens te compenseren. Uit de evaluatie van het platform is gebleken dat de huidige hardware over onvoldoende rekenkracht beschikt om met hoge snelheid en hoge nauwkeurigheid drift te detecteren. De zware berekeningen die hiervoor nodig zijn, zijn om die reden verlegd naar de PC. De features waaruit de optical flow vectoren worden berekend, worden oorspronkelijk geselecteerd met de FAST detector. Omdat rekenkracht niet langer een probleem vormt, is de GFTT detector uitgeprobeerd, een andere, preciezere detector die meer rekenkracht eist. Na vergelijking met de FAST detector is gebleken dat GFTT duidelijk beter presteert, vooral wanneer de beelden, waarover optical flow berekend wordt, wazig zijn. Ik concludeer dat drift nu een stuk nauwkeuriger wordt gedetecteerd. Hierdoor kan drift dan ook beter gecompenseerd worden.

\npar Wanneer gebruik wordt gemaakt van een PID-regelaar moet het correctiesignaal op de juiste manier teruggevoerd worden. De hoogte wordt geregeld door de aangelegde throttle. Er is verondersteld dat er een ideale waarde voor deze throttle bestaat die ervoor zorgt dat het platform op \'e\'en hoogte blijft vliegen. Deze waarde is nodig om het correctiesignaal van de PID-regelaar bij op te tellen. De ideale throttlewaarde hangt echter af van de resterende batterijspanning. Hoe lager de spanning, hoe hoger deze waarde zal zijn. Als oplossing wordt tijdens het opstijgen een schatting gemaakt van deze ideale throttlewaarde. Dit door de versnelling van het platform in z-richting te meten. Om op een vaste hoogte te kunnen vliegen moet de fout op de schatting nog gecompenseerd worden door de integrerende term van de PID-regelaar. Ik besluit dat voor een goeie hoogteregeling vanaf het begin van de vlucht een goede schatting van de ideale throttlewaarde essentieel is. Door optimale PID-parameters te gebruiken, is de hoogteregeling in het vervolg van de vlucht ook beter.

\npar Er is aangetoond dat een goede hoogteregeling essentieel is indien voor driftdetectie optical flow wordt gebruikt. Wanneer het platform stijgt of daalt, kan met optical flow drift gedetecteerd worden die er eigenlijk niet is. Om dit probleem aan te pakken, wordt gebruik gemaakt van een functiegenerator die de gewenste hoogte gestaag naar de eigenlijke setpoint laat convergeren. Omdat de hoogteveranderingen nu allemaal trager verlopen, is de valse drift die optical flow detecteert een stuk kleiner. De compensatie die het platform uitvoert om deze valse drift tegen te gaan is dan ook kleiner. Hierdoor is het platform dan weer wat stabieler. Bovendien zorgt het gebruik van de functiegenerator er ook voor dat de hoogte van het platform minder overshoot heeft. Dit is te verklaren door het feit dat de functiegenerator het platform een maximale snelheid oplegt waarmee het mag stijgen of dalen. Ik concludeer dat het gebruik van de functiegenerator zowel op vlak van hoogte als op vlak van positie bijdraagt aan een stabieler vlieggedrag.

\npar De positieregelaar is aangepast en vervolgens ook geoptimaliseerd. Er wordt gekozen voor een PI-regelaar die de snelheid van het platform regelt. Wanneer het platform op eenzelfde plaats moet blijven vliegen, wordt de gewenste snelheid op nul gehouden. Wanneer een nieuwe positie bereikt moet worden kan de snelheid in de richting van deze nieuwe positie aangepast worden. Ik besluit dat dit een goeie aanpak is om de positie van het platform te regelen.

\npar Mijn conclusie is dat de combinatie van bovenstaande optimalisaties het platform voldoende stabiel maakt om over te gaan naar het tweede deel van deze thesis, het in kaart brengen van de omgeving van het platform.

\section{Omgevingsmapping} \label{sec:concSLAM}
\npar Het tweede deel van deze thesis gaat over omgevingsmapping met SLAM. Er is op zoek gegaan naar een geschikt SLAM algoritme om aan omgevingsmapping te doen. Hiervoor zijn twee paradigma's met elkaar vergeleken. SLAM algoritmes met Extended Kalman filters en algoritmes met Particle filters. Omwille van de beperkte rekenkracht op het platform, is voor het laatste van de twee gekozen. Binnen dit paradigma zijn nog steeds heel veel verschillende implementaties mogelijk. Als eerste is {HectorSLAM} geprobeerd, dit algoritme houdt rekening met alle zes vrijheidsgraden van het platform. {HectorSLAM} vereist het gebruik van een ROS raamwerk. Omdat het niet is gelukt om dit werkende te krijgen, is er verder gezocht naar een andere implementatie. Vervolgens is er geprobeerd om {TinySLAM} te implementeren. Dit is een vrij licht SLAM algoritme dat uit een heel beperkt aantal lijnen code bestaat. Het is echter niet gelukt om {TinySLAM} te implementeren omdat de interface van dit algoritme nogal cryptisch geschreven is. Uiteindelijk is er voor {BreezySLAM} gekozen. Dit algoritme is gebaseerd op {TinySLAM}, maar heeft wel een goeie interface.

\npar Om te beginnen is {BreezySLAM} in de architectuur van het platform ge\"integreerd. {BreezySLAM} kan zowel met als zonder odometrie gebruikt worden. De prestaties van deze twee mogelijkheden zijn dan ook met elkaar vergeleken. Uit dit onderzoek blijkt dat {BreezySLAM} goed presteert zolang de te verkennen ruimte klein blijft en eenzelfde plaats niet meer dan eens wordt bezocht. In de praktijk is bij het verkennen van een ruimte het opnieuw bezoeken van eenzelfde plaats vaak onvermijdelijk. Ik kan concluderen dat het gekozen SLAM algoritme niet ideaal is om onbekende ruimtes te verkennen. Desondanks is er wel bewezen dat omgevingsmapping met het platform mogelijk is.

\npar Mijn conclusie is dat omgevingsmapping met een low-cost quadcopter zeker kan. Met goedkope hardware kunnen voldoende nauwkeurige mappen gemaakt worden van kleine gebouwen. Voor grotere gebouwen, eist SLAM echter wel heel veel geheugen. Het is dan ook beter om op de quadcopter enkel een lokale map bij te houden, terwijl een globale map extern wordt gemaakt. De lokale map kan gebruikt worden door de quadcopter voor het ontwijken van obstakels, de globale map kan dan gebruikt voor routeplanning op een hoger niveau. 

\npar Quadcopters zullen in de toekomst ongetwijfeld ingezet worden voor search and rescue missies. In tegenstelling tot grondrobots die niet geschikt zijn voor ruw terrein, kunnen quadcopters overal vliegen. Bij reddingsoperaties kan een nauwkeurige 3D map van een volledig gebouw van onschatbare waarde zijn. Hoe sneller deze map er is, hoe beter. Het lijkt dan ook logisch dat in de toekomst meerdere quadcopters zullen samenwerken om een gebouw nog sneller in kaart te brengen.

\section{Toekomstig werk}
Het platform is aan een update toe. Een aantal crashes hebben het nogal beschadigd. Het frame is wat geplooid, het centrum voor de elektronica moest hier en daar samen gelijmd worden. Er wordt aangeraden om een nieuw frame te ontwerpen, met een meer robuuste kern. De lijst van hardwarecomponenten die erop moeten is gekend. Het is dus enkel nog een kwestie om overal genoeg plaats te voorzien en ervoor te zorgen dat alles met vijzen bevestigd kan worden. Als het frame volledig recht zou zijn, dan zal het toestel in staat zijn mooi verticaal op te stijgen, in deze thesis is dit niet het geval en is dit ook de oorzaak geweest van meerdere crashes.

\npar De gebruikte autopilot, de APM 1.4 is verouderd. Tegenwoordig is de APM 2.6 reeds beschikbaar. Deze autopilot zorgt sowieso al voor een stuk stabieler vlieggedrag omdat er minder drift op zijn sensoren zit. Bovendien kan er ook een optical flow chip aan verbonden worden, waardoor de RPi veel meer rekenkracht overhoudt voor het verkennen en mappen van de omgeving.

\npar Op vlak van software wordt er aangeraden over te schakelen naar het ROS raamwerk. {HectorSLAM} kan dan gebruikt worden voor omgevingsmapping. Dit algoritme houdt rekening met alle zes vrijheidsgraden van het platform. Als het CPU-gebruik dit toelaat kan zelfs nog overgestapt worden naar een 3D SLAM algoritme zoals {OctoSLAM} \cite{paper:OctoSLAM}. Als dit niet blijkt te lukken, kan de laserscanner ook aan een lagere frequentie gebruikt worden dan zijn maximumfrequentie \SI{10}{\Hz}.